% 自訂字體的封包
\usepackage{fontspec} 
\setmonofont{CodeNewRomanNerdFontMono-Regular.otf}[Scale=MatchLowercase]
\usepackage{color} 
\usepackage{xcolor}



% 數學工具及符號
\usepackage{mathtools, amsmath, amsfonts, amsthm, latexsym} 

% 定义一个新的计数器
% 导入必要的包
\usepackage{xcolor}
\usepackage[most]{tcolorbox} % 用于创建带有阴影效果的框
\definecolor{myblockbgcolor}{RGB}{241, 241, 255}
\definecolor{Myblue}{RGB}{241, 241, 255}
\definecolor{Default_Blue}{RGB}{52,51,171}
\setbeamercolor{structure}{fg=Myblue} 
% 定义一个新的计数器
\newcounter{defname}

% 自定义问题环境
\newenvironment{mydef}{
  \stepcounter{defname}
  \begin{tcolorbox}[colframe=black, colback=blue!10, title=\textbf{定义 \arabic{defname}}]
}{
  \end{tcolorbox}
}

% 公式
% 自定义tcolorbox代码框样式
% 自定义tcolorbox代码框样式
\newtcolorbox[auto counter, number within=section]{listing}[2][]{%
  colback=yellow!10!white,      % 代码背景色
  colframe=black,                % 代码框框架颜色为黑色
  title=\textbf{代码},           % 标题为加粗的"代码"
  fonttitle=\bfseries           % 标题加粗
}


\usepackage{xcolor} % 如果你需要自定义颜色

% 分別將數學符號間的間隔加大及加粗
\usepackage{newtxtext,newtxmath}

% 圖表自動編號的封包
\usepackage{caption} 

%% 設定自動編號
\setbeamertemplate{caption}[numbered]

% 導入圖形與表格的封包
\usepackage{graphicx}  % \scalebox{} 可用於將過大的表格縮小
\usepackage{booktabs}

% 排列多個子圖形的封包
\usepackage{subfigure} 

% 允許表格的一格能多列呈現的封包
\usepackage{multirow} 

% 可指定表格排版的封包
\usepackage{array}

% 翻轉表格的封包
\usepackage{lscape} 

% 序列標號
\usepackage{enumerate} 

% 繪圖封包 (用於添加浮水印)
\usepackage{tikz}
\usepackage{svg}
\usepackage{float}

% 引注參考資料
\usepackage{natbib}

% 註釋掉大部分的封包
\usepackage{comment}

% 浮水印

% 设置背景模板
\usebackgroundtemplate{%
    \tikz[overlay, remember picture] % 使用 TikZ 创建一个覆盖层
        \node[opacity=0.1, anchor=center, at=(current page.center)] % 设置透明度和位置
            {\includegraphics[scale=0.0618]{Fig/上海财经大学-logo-2048px.png}}; % 载入并缩放图片
}

\mode<presentation>{
%\usetheme{default}
%\usetheme{AnnArbor}
%\usetheme{Antibes}
%\usetheme{Bergen}
%\usetheme{Berkeley}
%\usetheme{Berlin}
% \usetheme{Boadilla}
% \usetheme{CambridgeUS}
% \usetheme{Copenhagen}
%\usetheme{Darmstadt}
%\usetheme{Dresden}
%\usetheme{Frankfurt}
%\usetheme{Goettingen}
%\usetheme{Hannover}
%\usetheme{Ilmenau}
%\usetheme{JuanLesPins}
%\usetheme{Luebeck}
\usetheme{Madrid}
%\usetheme{Malmoe}
%\usetheme{Marburg}
%\usetheme{Montpellier}
%\usetheme{PaloAlto}
%\usetheme{Pittsburgh}
%\usetheme{Rochester}
%\usetheme{Singapore}
%\usetheme{Szeged}
%\usetheme{Warsaw}
\usetheme[style=light]{Nord}

%----------------------------------------------------------------------------------------
%	外框形式 (擇一,不選等同選擇默認的外框形式)
%----------------------------------------------------------------------------------------

\useoutertheme{default}
% \useoutertheme{infolines}
% \useoutertheme{miniframes}
% \useoutertheme{smoothbars}
%\useoutertheme{sidebar}
%\useoutertheme{split}
% \useoutertheme{shadow}
% \useoutertheme{tree}
% \useoutertheme{smoothtree}

%----------------------------------------------------------------------------------------
%	外框的自訂義調整 
%----------------------------------------------------------------------------------------

% 外框上緣的字 (fg) 為黑色,背景 (bg) 為白色。
%\setbeamercolor{section in head/foot}{fg=white, bg=black} 

% 外框上緣顯示的章節(section)頁數標籤是否關閉
% \setbeamertemplate{mini frames}{}   

% 調整外框形式的字體大小
\setbeamerfont{headline}{size=\scriptsize}
\setbeamerfont{footline}{size=\scriptsize}

% 取消右下方的跳轉工具列
%\setbeamertemplate{navigation symbols}{} 

% % 自定義1:外框下緣僅出現名字及頁碼
% \setbeamertemplate{footline}
% {\leavevmode%
% \hbox{%
% \begin{beamercolorbox}[wd=0.5\paperwidth,ht=3ex,dp=1ex,leftskip=2ex]%
% {author in head/foot}%
% {\footnotesize\textbf{\insertshortauthor}}%
% \end{beamercolorbox}%
% \begin{beamercolorbox}[wd=0.5\paperwidth,ht=3ex,dp=1ex,right]%
% {author in head/foot}%
% \footnotesize \textbf{{\insertframenumber{} / \inserttotalframenumber\hspace*{2ex}}} %頁碼控制選項
% \end{beamercolorbox}%
% }}

%% 自定義2:清除外框下緣但僅出頁碼
%\setbeamertemplate{footline}[page number] 

%% 自定義3:清除外框下緣
%\setbeamertemplate{footline}[] 

%----------------------------------------------------------------------------------------
%	顏色主題 (擇一,不選等同選擇默認的顏色主題)
%----------------------------------------------------------------------------------------

% \usecolortheme{default}
%\usecolortheme{albatross}
% \usecolortheme{beaver}
%\usecolortheme{beetle}
%\usecolortheme{crane}
% \usecolortheme{dolphin}
\usecolortheme{dove}
% \usecolortheme{fly}
% \usecolortheme{lily}
%\usecolortheme{orchid}
 % \usecolortheme{rose}
%\usecolortheme{seagull}
%\usecolortheme{seahorse}
% \usecolortheme{whale}
%\usecolortheme{wolverine}
% \usecolortheme{spruce}
%----------------------------------------------------------------------------------------
%	顏色主題的自訂義調整 
%----------------------------------------------------------------------------------------

% 全文的主題色 (可以特別針對報告對象或機構的代表色調整!)
% \setbeamercolor{structure}{fg=Myblue} 

% 封面頁中標題區塊的底色及字體顏色
%\setbeamercolor{title}{bg=green, fg=black} 

% 各頁標題區塊的底色及字體顏色
%\setbeamercolor{frametitle}{bg=white,fg=black} 

% 全文的內文顏色
%\setbeamercolor{normal text}{fg=orange}

% 數學區塊的標題顏色 
%\setbeamercolor{block title}{bg=blue,fg=yellow} 

% 數學區塊的內文顏色 
%\setbeamercolor{block body}{bg=green,fg=red} 

% 警示文字的顏色
\setbeamercolor{alerted text}{fg=red} 

%----------------------------------------------------------------------------------------
%	enumerate 及 item 的形狀
%----------------------------------------------------------------------------------------

%\useinnertheme{rounded} % 圓球 (3D)
\useinnertheme{circles} % 圓形 (2D)
%\useinnertheme{rectangles} % 方形
% \useinnertheme{triangle} % 三角形
%\useinnertheme{inmargin} % 插入邊沿
\setbeamertemplate{itemize items}[triangle]

%----------------------------------------------------------------------------------------
%	自訂 item 的顏色
%----------------------------------------------------------------------------------------

%\setbeamercolor{item projected}{bg=red}

%----------------------------------------------------------------------------------------
%	個人化的設置及細節調整
%----------------------------------------------------------------------------------------

% 設定頁面邊界
\setbeamersize{text margin left=  0.75cm, text margin right= 0.75cm}
\special{papersize=\the\paperwidth,\the\paperheight}
\providecommand{\tabularnewline}{\\}
}

